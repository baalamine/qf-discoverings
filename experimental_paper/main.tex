\documentclass{vldb}
\usepackage{graphicx}
\usepackage{balance}  % for  \balance command ON LAST PAGE  (only there!)
\newcommand{\lamine}[1]{\textbf{[Lamine: \textcolor{blue}{#1}]}}

\begin{document}

\title{A Sample {\ttlit Proceedings of the VLDB Endowment} Paper in LaTeX
Format\titlenote{for use with vldb.cls}}


\numberofauthors{3} 
\author{
% 1st. author
\alignauthor
Dalia Attia Wagui\\
       \affaddr{}\\
       \affaddr{}\\
       \affaddr{}\\
       \email{}
% 2nd. author
\alignauthor
Mouhamadou Lamine BA\\
       \affaddr{Qatar Computing Research Institute}\\
       \affaddr{Tornado, Tower}\\
       \affaddr{18th floor, Doha, Qatar}\\
       \email{mlba@qf.org.qa}
% 3rd. author
\alignauthor 
Laure Berti-Equille\\
     \affaddr{Qatar Computing Research Institute}\\
       \affaddr{Tornado, Tower}\\
       \affaddr{18th floor, Doha, Qatar}\\
       \email{lberti.qf.org.qa}
}
% There's nothing stopping you putting the seventh, eighth, etc.
% author on the opening page (as the 'third row') but we ask,
% for aesthetic reasons that you place these 'additional authors'
% in the \additional authors block, viz.
%\additionalauthors{Additional authors: John Smith (The Th{\o}rv\"{a}ld Group, {\texttt{jsmith@affiliation.org}}), Julius P.~Kumquat
%(The \raggedright{Kumquat} Consortium, {\small \texttt{jpkumquat@consortium.net}}), and Ahmet Sacan (Drexel University, {\small \texttt{ahmetdevel@gmail.com}})}
%\date{30 July 1999}
% Just remember to make sure that the TOTAL number of authors
% is the number that will appear on the first page PLUS the
% number that will appear in the \additionalauthors section.


\maketitle

\begin{abstract}
\end{abstract}

\section{Introduction}
\section{Truth Finding Algorithms}

\lamine{Some new ideas to improve this section}
\begin{itemize}
 \item Review all the pseudo-codes of the different truth finding 
 algorithms and bring some corrections to some of them.
 \item Propose an hierarchical graph of studied truth finding algorithms
 by clustering them according to the similar considered features and the 
 derivation too.
 \item Propose a validation of the parameter setting of each algorithm
 inferred from tests on Book dataset by also using the other datasets
 for verufication wether or not these parameters remain still efficient.
\end{itemize}
\end{document}
