\documentclass{vldb}
\usepackage{graphicx}
\usepackage{balance}  % for  \balance command ON LAST PAGE  (only there!)
%\newcommand{\lamine}[1]{\textbf{[Lamine: \textcolor{blue}{#1}]}}

\begin{document}
\title{Truth Finding: An Experimental Study of Robustness and Scalability of Algorithms}
\numberofauthors{3} 
\author{
% 1st. author
\alignauthor
Dalia Attia Wagui\\
 %      \affaddr{XXXX}\\
  %     \affaddr{YYYY}\\
   %    \affaddr{ZZZZ}\\
    %   \email{aaaaaa}
% 2nd. author
\alignauthor
Mouhamadou Lamine BA\\
     %  \affaddr{Qatar Computing Research Institute}\\
      % \affaddr{Tornado, Tower}\\
       %\affaddr{18th floor, Doha, Qatar}\\
       %\email{mlba@qf.org.qa}
% 3rd. author
\alignauthor 
Laure Berti-Equille\\
     %\affaddr{Qatar Computing Research Institute}\\
      % \affaddr{Tornado, Tower}\\
      % \affaddr{18th floor, Doha, Qatar}\\
       %\email{lberti.qf.org.qa}
}%
% There's nothing stopping you putting the seventh, eighth, etc.
% author on the opening page (as the 'third row') but we ask,
% for aesthetic reasons that you place these 'additional authors'
% in the \additional authors block, viz.
%\additionalauthors{Additional authors: John Smith (The Th{\o}rv\"{a}ld Group, {\texttt{jsmith@affiliation.org}}), Julius P.~Kumquat
%(The \raggedright{Kumquat} Consortium, {\small \texttt{jpkumquat@consortium.net}}), and Ahmet Sacan (Drexel University, {\small \texttt{ahmetdevel@gmail.com}})}
%\date{30 July 1999}
% Just remember to make sure that the TOTAL number of authors
% is the number that will appear on the first page PLUS the
% number that will appear in the \additionalauthors section.
\maketitle
\begin{abstract}
\paragraph*{Motivation et Outline} Truth finding is an important because conflicting , erroneous, and dirty
information are everywhere. The truth must be tell when reconciling such a conflicting data from different sources. 
This has lead to much effort of the database community and well founded truth discovering algorithms. However, there 
is a lack of a comparative study of both the scalability and the robustness of these algorithms. The existing comparative 
studies only focus on accuracy aspects. We describe, reimplement, and compare the most prominent solutions, so far, 
for the truth finding problems. To tackle this lack, we propose in this paper an experimental study of the robustness
and the scalability of the most referenced truth finding algorithms.  Our outline is as follows.


First, we overview the truth finding problem by giving preliminary definitions, a classificartion of the literature, and
by describe in details the algorithms (most referenced algorithms) we have considered in this study.


\end{abstract}

\section{Introduction}
\section{Overview on Truth Finding Algorithms}
\subsection{Preliminaries}
\subsection{Classification}
Cluster the algorithms in the three following classes.
\begin{itemize}
 \item Iterative algorithms:
 \item EM based algorithms:
 \item Dependency detection based algorithms:
\end{itemize}
\subsection{Twelve Truth Finding Algorithms}
We details in the sections, the truth finding 
algorithms we have considered for the comparative study.
Some points to put in this section.
\begin{itemize}
 \item Pseudo-code of each algorithm
 \item Hierarchical graph based representation of the common characteristics
 of the different algorithms
 \item Validation of parameter setting inferred from experiments on the Book
 dataset by using other datasets, e.g., Flight dataset. (Maybe this point should
 be moved to the experimental secion).
\end{itemize}

\section{Experimental Setting}

\begin{itemize}
 \item Propose guidance  on the algorithms from the results of the experiments
\end{itemize}
\end{document}
