\documentclass[a4paper,10pt]{scrartcl}
\usepackage[utf8]{inputenc}
\usepackage{hyperref}
\renewcommand{\rmdefault}{ptm}
\usepackage{bera}% optional: just to have a nice mono-spaced font
\usepackage{listings}
\usepackage{xcolor}
\usepackage{graphicx}
\usepackage{filemod}
\colorlet{punct}{red!60!black}
\definecolor{background}{HTML}{EEEEEE}
\definecolor{delim}{RGB}{20,105,176}
\colorlet{numb}{magenta!60!black}

\lstdefinelanguage{json}{
    basicstyle=\normalfont\ttfamily,
    numbers=left,
    numberstyle=\scriptsize,
    stepnumber=1,
    numbersep=8pt,
    showstringspaces=false,
    breaklines=true,
    frame=lines,
    backgroundcolor=\color{background},
    literate=
     *{0}{{{\color{numb}0}}}{1}
      {1}{{{\color{numb}1}}}{1}
      {2}{{{\color{numb}2}}}{1}
      {3}{{{\color{numb}3}}}{1}
      {4}{{{\color{numb}4}}}{1}
      {5}{{{\color{numb}5}}}{1}
      {6}{{{\color{numb}6}}}{1}
      {7}{{{\color{numb}7}}}{1}
      {8}{{{\color{numb}8}}}{1}
      {9}{{{\color{numb}9}}}{1}
      {:}{{{\color{punct}{:}}}}{1}
      {,}{{{\color{punct}{,}}}}{1}
      {\{}{{{\color{delim}{\{}}}}{1}
      {\}}{{{\color{delim}{\}}}}}{1}
      {[}{{{\color{delim}{[}}}}{1}
      {]}{{{\color{delim}{]}}}}{1},
}
\newcommand{\shellcmd}[1]{\vspace*{1cm}\\\indent\indent\texttt{\# #1}\vspace*{1cm}}
\newcommand{\claim}[1]{\vspace*{0.5cm}\\\indent\indent\texttt{#1}\vspace*{0.5cm}}

%opening
\title{SYNTHETIC DATASET GENERATOR FOR TRUTH DISCOVERY SCENARIOS}
\subtitle{Documentation}
\author{Mouhamadou Lamine BA and Laure BERTI-EQUILLE}
\date{\begin{tabular}{ll}Creation date~:& November 5, 2015 \\ Revision date~:& \today \\\end{tabular}}

\begin{document}

\maketitle
\newpage
% tableofcontents
\tableofcontents
\newpage

% Introduction
\section{Introduction}
In order to provide to users the ability to test and compare multiple truth finding algorithms on various data set settings, some of them 
being not easy to obtain in practice, we provide the \href{http://daqcri.github.io/dafna/#/dafna/apidoc/gettingstarted.html}{DAFNA API} 
together with a synthetic dataset generator. The synthetic dataset generator allows to simulate a large variety of scenarios where sources
present different behaviors in terms of coverage, error rate, reliability level, conflicting information, ect. More importantly, the generator
guarantees a ground truth for the evaluation of the accuracy of compared truth finding algorithms. 
% Free Jar
\section{JAVA Set Up}
The synthetic dataset generator has been implemented in JAVA. The correspond jar file is made avaible for free download on \url{http://daqcri.github.io/dafna/#/dafna/exp\_sections/realworldDS/synthetic/syntheticDs.html}
with the name DAFNA-DataSetGenerator.jar. Please verify you have a recent version of JAVA JRE installed in your computer before using the jar file.

% Types of datasets
\section{Types of datasets}
The synthetic dataset generator can produce various types of datasets in function of
the excepted source coverage, ground truth distribution, distinct values distribution, 
and level of similarity between distinct of the same data items. It also offers the ability 
to capture both extreme optimistic and pessimistic scenarios by playing on ground truth distribution 
controller.

% Inputs parameters
\section{Generator Usage}
The synthetic dataset generator is used by running the following command line, e.g.
on a UNIX terminal.
\shellcmd{java -jar DAFNA-DataSetGenerator.jar param\_1 param\_2 \ldots param\_n}

\noindent param\_1, param\_2, \ldots, param\_1 correspond to the different input parameters
of the generator.



% Mandatory parameters
\subsection{Input parameters}
The synthetic dataset generator considers as input 10 parameters.
All parameters are mandaratory for a successful generation. Each 
parameter controls a given characteristic of the datasets to be generated.
Below details the semantics of each parameter.

\begin{itemize}
 \item[\textbf{-src}:] The number of sources providing claims
 \item[\textbf{-obj}:] The number of objects covered by sources
 \item[\textbf{-prop}:] The number of properties describing each object
 \item[\textbf{-cov}:] Value from $0$ to $1$ representing the uniform percentage 
 of data items covered by sources in the Uniform coverage Distribution.
 \item[\textbf{-ctrlC}:] The source coverage distribution which is either \textbf{uniform} or \textbf{Exp} (i.e. Random).
 An uniform distribution means that the number of values provided by the sources is uniformly
 distributed given the source coverage value. In the same spirit, the number of values provided 
 by the sources is exponentially distributed when its coverage distribution is exponential.
 \item[\textbf{-ctrlT}:] The Ground Truth Distribution per Source. It can be set to one of the following distribution:
 \begin{itemize}
  \item[R:] Random Distribution in which the number of true positive claims per source is randomly chosen.
  \item[Uniform:] Uniform Distribution where each source provides the same number of true positive claims.
  \item[FP:] Fully Pessimistic Distribution which considers that 80\% of the sources provide always false claims
  while 20\% of the sources provide always true positive claims.
  \item[FO:] Fully Optimistic Distribution which considers that 80\% of the sources provide always true claims
  while 20\% of the sources provide always false positive claims.
  \item[80P:] 80\% Pessimistic Distribution in which 80\% of the sources provide 20\% true positive claims whereas
  20\% of the sources provide 80\% false positive claims. 
  \item[80O:] 80\% Optimistic Distribution which considers that 80\% of the sources provide 80\% true positive claims 
  and only 20\% of the sources provide 20\% true positive claims.
  \item[Exp] Exponential Distribution in which the number of true positive values provided by the sources is exponentially
  distributed.
 \end{itemize}
 \item[\textbf{-v}:] The number of distinct value per data item. The specified value will be used as a constant for the uniform 
 model, and as the maximum number of distinct values in the exponential model.
 \item[\textbf{-ctrlV}:] The distinct values distribution per data item. It can be set either uniform  or exponential.
 \begin{itemize}
  \item [Uniform:] All data items have the same number of distinct values claimed by the set of sources.
  \item [Exp:]  Each data item has a number of distinct values that is exponentially distributed.
 \end{itemize}
  \item[\textbf{-s}:] Similarity level between the different values of the same data item. It must be set either to \textbf{sim} or 
 to \textbf{dissSim}.
  \begin{itemize}
   \item [Sim] Distinct values for data items will be highly similar.
   \item [dissSim] Distinct values for data items will be highly diss-similar.
  \end{itemize}
 \item[\textbf{-f}:] The output folder where the data set will be created and saved.
\end{itemize}

% generator output
\subsection{Output Datasets}
Given the required input parameters, the synthetic dataset generator proceeds to the generation 
of the dataset with the desired characteristics. After completion, it creates the two following
folders into the main directotry specified by the user with the parameter \textbf{-f}.
\begin{itemize}
 \item[claims/] Folder which contains the dataset file 
 \item[truth/]  Folder which contains the ground truth file
\end{itemize}

The dataset file produced by the generator contains the sources together with the set of claims.
It corresponds to a csv file in which each line	contains the following fields:
\claim{ClaimId, Object, Property, Claim, Source, TimeStamp}

\noindent where claimId, Object, Property, Claim, Source, and TimeStamp are respectively the identifer of the claim,
the real world object, one of its properties, a claim about this property, the source who made the claim, and
the time when the claim has been made.

The ground truth file contains the ground truth associated to the generated dataset. It is also a csv file in which each line
contains the following fields:
\claim{Object, property, Trueclaim}
where Object, property, and TrueClaim represent respectively an object, one among its set of properties, and the true claim about
this given property.

% Example of generation
\section{Example of generation}
The use of the synthetic dataset generator requires to instanciate all the  parameters in a certain order, as we will see in the next.
As an example of a generation, suppose we would like to obtain a dataset consisting of 10 sources which fully cover  10 objects having 5 
properties. We also want to constrain the distribution of the distinct values per data item and the ground truth distribution to be both exponential.
We do not care about similar distinct values. Finally, we would like to set the default directotry for the output as ``./Test". 
We obtain the needed configuration by running the synthetic dataset generator as follows.
\shellcmd{java -jar DAFNA-DataSetGenerator.jar -src 10 -obj 10 -prop 5 -cov 1.00 -ctrlC Exp -ctrlT Exp -v 3 -ctrlV Exp -s dissSim -f "./Test"}

Figures~(a) and (b) respectively show an excerpt of the dataset file and the ground truth file produced by the generator after completion.\\


\begin{tabular}{cc}
\includegraphics[height=0.4\textheight]{claims}&\includegraphics[height=0.4\textheight]{truths}\\
(a) Excerpt of the dataset file & (b) Excerpt of the ground truth File\\
\end{tabular}

\end{document}
