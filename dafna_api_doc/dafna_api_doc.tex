\documentclass[a4paper,10pt]{scrartcl}
\usepackage[utf8]{inputenc}
\usepackage{hyperref}
\renewcommand{\rmdefault}{ptm}
\usepackage{bera}% optional: just to have a nice mono-spaced font
\usepackage{listings}
\usepackage{xcolor}

\colorlet{punct}{red!60!black}
\definecolor{background}{HTML}{EEEEEE}
\definecolor{delim}{RGB}{20,105,176}
\colorlet{numb}{magenta!60!black}

\lstdefinelanguage{json}{
    basicstyle=\normalfont\ttfamily,
    numbers=left,
    numberstyle=\scriptsize,
    stepnumber=1,
    numbersep=8pt,
    showstringspaces=false,
    breaklines=true,
    frame=lines,
    backgroundcolor=\color{background},
    literate=
     *{0}{{{\color{numb}0}}}{1}
      {1}{{{\color{numb}1}}}{1}
      {2}{{{\color{numb}2}}}{1}
      {3}{{{\color{numb}3}}}{1}
      {4}{{{\color{numb}4}}}{1}
      {5}{{{\color{numb}5}}}{1}
      {6}{{{\color{numb}6}}}{1}
      {7}{{{\color{numb}7}}}{1}
      {8}{{{\color{numb}8}}}{1}
      {9}{{{\color{numb}9}}}{1}
      {:}{{{\color{punct}{:}}}}{1}
      {,}{{{\color{punct}{,}}}}{1}
      {\{}{{{\color{delim}{\{}}}}{1}
      {\}}{{{\color{delim}{\}}}}}{1}
      {[}{{{\color{delim}{[}}}}{1}
      {]}{{{\color{delim}{]}}}}{1},
}
\newcommand{\lamine}[1]{\textcolor{blue}{#1}}
%opening
\title{DAFNA -- RestFul API for Truth Discovery Algorithms}
\subtitle{API DOCUMENTATION}
\author{Mouhamadou Lamine Ba\\ Hossam. H. HAMMADY\\ Laure BERTI-EQUILLE}
\date{\begin{tabular}{ll}Creation date:&\today\end{tabular}}

\begin{document}

\maketitle
\newpage
% tableofcontents
\tableofcontents
\newpage

% Introduction
\section{Introduction}
Data Forensics with Analytics, or for short DAFNA, is a project initiated by the Data Analytics Group in Qatar Computing Research Institute, 
Hamad Bin Khalifa University.
One of its various facets aims at providing efficient algorithms and tools for verifying the veracity of information provided by multiple conflicting
sources. The DAFNA API for truth discovery is a RestFul one that enables end user applications to access to twelve existing truth discovery algorithms
for fact checking and truth discovery when they have to face the integration of information coming from untrustworthy sources. The present documentation 
goes in deep on the use of this API.


% Authentication
\section{Authentication}
Users who would like to use DAFNA API, or integrate it in their own applications,
need to first get the necessary credentials to access the API.
DAFNA uses the standard
\href{http://oauth.net/2/}{oAuth2} protocol to authenticate 3rd party 
API clients on behalf of the end users. Every 3rd party developper must request
for an access token, consisting of
a pair of API Access Key and Secret Key from DAFNA developpers by contacting
\href{dafna@qcri.org}{dafna@qcri.org} after registration on \href{http://dafna.qcri.org/allegatortrack}{http://dafna.qcri.org/allegatortrack}.
The 3rd party client will receive from DAFNA developpers the following access credentials.
\begin{itemize}
 \item Authentication token: API user's Access Key
 \item Encrypted password: User's Secret Key
\end{itemize}
\lamine{Is the description of the authentication phase correct?}
% Request format
\section{Request Format}
% Scheme
\subsection{Scheme}
Requests to the API endpoints are done using plain HTTP. While HTTPS requests are not yet
supported by the API, it should do in the future to guarantee a more secure transmission of
the Access Token. However, supporting HTTPS might yield requests/responses little bit slower.

\lamine{Is it right that https is not yet supported by the API?}
% Methods
\subsection{Methods}
The current version of DAFNA API supports three HTTP Requests, i.e., \textbf{GET}, \textbf{POST}, and \textbf{DELETE}, 
with respect to the operations, for instance creation, listing, or removal of datasets, the 3rd party applications would
like to perform on the behalf on the client.
% Base Url
\subsection{Base URLs}
\lamine{What's the main difference between the two base URLs? In which scenario, each of the two should be used? Production or development?}
\begin{itemize}
 \item \href{http://dafna.qrci.org/}{http://dafna.qrci.org/}
 \item \href{http://dafna.qcri.org/}{http://dafna.qcri.org/}
\end{itemize}
All the endpoint provided by the API are appended to the base URLs.
To respectively create and list the datasets, one must use the following URLs with some parameters
as we will see in the next. 
\begin{description}
\item GET \href{http://dafna.qcri.org/datasets/}{http://dafna.qcri.org/datasets/} \\
 \item POST \href{http://dafna.qcri.org/datasets/}{http://localhost:4000/datasets/}
\end{description}
% Request content type
\subsection{Request Content Type}
Either (1) ``application/x-www-form-urlencoded” or (2) ``application/json” can be used as the request content type (HTTP request “Content-Type” header value). 
If (1) is used, all parameters should be encoded in the URL as a query string with proper escapes and separated by \&. If (2) is used, the request body should 
be a well-formed JSON object. 

\begin{description}
 \item url-encoded form:
 \item POST \href{http://localhost:4000/runsets?checked\_algo=array\_of\_parameters}{http://dafna.qcri.org/runsets?checked\_algo=array\_of\_parameters}
 \item json formatting:
 \item POST \href{http://localhost:4000/runsets}{http://dafna.qcri.org/runsets}
 \item \{checked\_algo: array\_of\_parameters, general\_config: array\_numerical\_values\}
\end{description}


Note that the method used to pass the parameters of a given request is somehow dictated by the manner the request is issued.
For instance, if a given 3rd party client uses
\href{http://curl.haxx.se/}{'Curl'} to issue his requests, the following two syntaxes are both valid.
\begin{description}
 \item curl -X DELETE http://localhost:4000/runsets/1/ -d ``user\_token=some\_string\_here"\\
 \item and \\
 \item curl -X POST http://localhost:4000/runsets -H ``Content-type: application/json" -d `\{``user\_token": ``some\_string\_here" , ``checked\_algo": \{``Accu": [``0.2", ``0", ``100", ``0.5", ``false", ``true", ``true", ``false"] \}, ``general\_config":[``0.001", ``0.8", ``1", ``0.4"]\}' 
\end{description}
% Response format
\section{Response Format}
% http status code
\subsection{HTTP Status Codes}
\begin{itemize}
 \item \textbf{200:} Successful Requests (otherwise, an internal error will occur)
 \item \textbf{401:} Unauthorized Access, e.g., missing, invalid, or expired token
 \item \textbf{403:} Forbidden Access to the resource requested by the client
 \item \textbf{404:} Not Found Resource requested by the user
 \item \textbf{422:} Invalid Action on the accessed resource
 
\end{itemize}

% Response body format
\subsection{Response Body Format}
Currently, all responses are returned in JSON format. In the future, more formats could be 
supported (like XML). The requested format can be explicitly specified in the request, but if omitted,
JSON will be assumed. The example above can be written as:
\begin{description}
 \item GET \href{http://dafna.qcri.org/datasets.json?user\_token=some\_string\_here}{http://dafna.qcri.org/datasets.json?user\_token=some\_string\_here}
\end{description}

% endpoints 
\section{Endpoints}
The current version of the API provides 8 endpoints, 4 among them are accessed via GET, 2 via POST, and 2 via DELETE:
% post endpoints
\subsection{POST Endpoints}
\begin{enumerate}
 \item \textbf{/datasets}
\begin{description}
 \item Requests the creation of a Dataset
\end{description}
\textbf{Parameters}
\begin{itemize}
\item \textbf{user\_token}: Secret Token provided to client for authentication 
\item \textbf{s3\_key}: your file's s3 key if uploaded at Amazon's s3
\item \textbf{other\_url}: URL if dataset file is located somewhere else
\item \textbf{kind}: claims, truth (type of dataset)
\item \textbf{original\_filename}: name of the file 
\end{itemize}
\textbf{Response}
\begin{description}
 \item None. It is enough to check for HTTP status code of 200 for success.	
\end{description}
\textbf{Notes}
\begin{description}
 \item The \textbf{s3\_key} and \textbf{other\_url} parameters are both optional.
\end{description}
%
 \item \textbf{/runsets}
 \begin{description}
 \item Creating a Runset, i.e., performing truth discovering over a given dataset
\end{description}
\textbf{Parameters}
\begin{itemize}
 \item \textbf{user\_token}: Secret Token provided to client for authentication 
 \item  \textbf{checked\_algo}: Algorithm name(s) followed by respective arrays of parameters. E.g. ``checked\_algo": \{``Accu": [``0.2", ``0", ``100", ``0.5", ``false", ``true", ``true", ``false"] , ``Cosine": [``1", ``0.2"] \}
 \item \textbf{general\_config}: an array of numerical values, e.g. [``0.001", ``0.8", ``1", ``0.4"]
\end{itemize}
\textbf{Response}
\begin{description}
 \item None. It is enough to check for HTTP status code of 200 for success.	
\end{description}
\end{enumerate}

% get endpoints
\subsection{GET Endpoints}
\begin{enumerate}
 \item \textbf{/datasets}
\begin{description}
 \item Listing Datasets 
\end{description}
\textbf{Parameters}
\begin{itemize}
\item \textbf{user\_token}: Secret Token provided to client for authentication
\item \textbf{kind}: either claims or truth (the type of requested datasets)
\item \textsf{start} 
\item \textsf{length}
\end{itemize}
\textbf{Response}
\begin{description}
 \item \{DS1, DS22, ...\} where the format of DSn is:\\
 \{
  draw: $<$draw$>$,
  recordsTotal: $<$Number of Records$>$,
  recordsFiltered: $<$Number of filtered Records$>$,
  data: $<$data$>$,
  s3\_direct\_post: $<$date\_time$>$,
  fields: $<$date\_time$>$,
  key: $<$date\_tim$e>$,
  policy: $<$policy$>$,
  signature: $<$signatur$>$,
  success\_action\_status: $<$status$>$,
\}
\end{description}
\textsf{Notes}
\begin{description}
 \item \textsf{start} and \textsf{end} parameters are optional. Note that 
 when the parameter \textsf{kind} is not provided, the user will get an empty 
 set of datasets.
\end{description}

%
\item \textbf{/runsets}
\begin{description}
 \item Listing Runsets
\end{description}
\textbf{Parameters}
\begin{itemize}
 \item \textbf{user\_token}: Secret Token provided to client for authentication
\end{itemize}
\textbf{Response:}
\begin{description}
 \item \{RS1, RS22, ...\} where the format of RSn is:\\
\{
  id: $<$runset id$>$,
  created\_at: $<$Time of creation$>$,
  runs:[ 
\{
  id: $<$runset id$>$,
  algo: $<$Time of creation$>$,
  crated\_at: $<>$,
  status: $<$whether finished$>$,
  duration: $<$Duration of run$>$,
\}
,\{\},\{\}...]
\}	
\end{description}
%
\item \textbf{/runsets/<id>/results}
\begin{description}
 \item Listing Results of runsets having id=<id>
\end{description}
\textbf{Parameters}
\begin{itemize}
 \item \textbf{user\_token}: Secret Token provided to client for authentication
 \item \textbf{id}: identier of the runsets
\end{itemize}
\textbf{Response}
\begin{description}
 \item
 \{
  draw: $<$draw$>$,
  recordsTotal: $<$Total Number of Records$>$,
  recordsFiltered: $<$Filtered Records$>$,
  data: 

\}
\end{description}
%
\item \textbf{/runs}
\begin{description}
\item Listing Runs
\end{description}
\textbf{Parameters}
\begin{itemize}
 \item \textbf{user\_token}: Secret Token provided to client for authentication
\end{itemize}
\textbf{Response}
\begin{description}
\item 
\{
  draw: $<$draw$>$,
  recordsTotal: $<$Total Number of Records$>$,
  recordsFiltered: $<$Filtered Records$>$,
  data:[ \{
  id: $<$run id$>$,
  algorithm: $<$Name of the algorithm$>$,
  created\_at: $<$Time of creation$>$,
  runset\_id: $<$Runset ID$>$,
  display: $<$Algorithm name and parameters in paranthesis (in order)$>$,
  status: $<$whether finished$>$,
  duration: $<$Duration of run$>$,
\}
,\{\},\{\}...]
\}	
\end{description}
\end{enumerate}

% post endpoints
\subsection{DELETE Endpoints}
\begin{enumerate}
\item \textbf{/datasets/<id>}
\begin{description}
 \item Deleting a Dataset having id=<id>
\end{description}
\textbf{Parameters}
\begin{itemize}
 \item \textbf{user\_token}: Secret Token provided to client for authentication
 \item \textbf{id}: identier of the dataset to be deleted 
\end{itemize}
\textbf{Response}
\begin{description}
 \item None. It is enough to check for HTTP status code of 200 for success.	
\end{description}
%
\item \textbf{/runsets/<id>}
\begin{description}
\item Deleting a Runset having id=<id>
\end{description}
\textbf{Parameters}
\begin{itemize}
  \item \textbf{user\_token}: Secret Token provided to client for authentication
 \item \textbf{id}: identier of the runset to be deleted 
\end{itemize}
\textbf{Response}
\begin{description}
\item  None. It is enough to check for HTTP status code of 200 for success.	
\end{description}
\end{enumerate}

% Job Monitoring 
\section{Pusher for Job Monitoring}
Some of the jobs you submit (for e.g. dataset upload, run algorithm(s)) take time to finish. 
To make the things easier for your application, we have a channel on \href{https://pusher.com/}{Pusher},
which your application can subscribe to and be notified for the events it has been binded to. If there is any notification 
(such as job completion), your applicaton will be notified of this. Please note that, repeatedly requesting
run statuses rather than listening to the push messages will eventually overwhelm our server, and we may choose
to suspend your account. So, please go through the documentation of Pusher to get a sense of how things work. Your 
application can subscribe to the notifications using app\_id = "79143". You need to contact us for a key. We will send 
you the key through email to enable you to subscribe to the channel. Once you have configured your application using the 
app\_id and key, you can bind to particular events, you wish to be notified for. for e.g. 

\begin{itemize}
 \item \textbf{channel.bind(`run\_change', function(run){}):} will notifiy in case of any change in the status of a run.
 \item \textbf{channel.bind(`dataset\_change', function(dataset){}):} will notify you in case of any change in the status of a dataset. 
\end{itemize}
You may find the following pages to be particularly useful:
\begin{enumerate}
 \item \href{https://pusher.com/docs/client_api_guide/client_channels}{Client Channel}
 \item \href{https://pusher.com/docs/client_api_guide/client_public_channels}{Client Public Channels}
 \item \href{https://pusher.com/docs/client_api_guide/client_connect}{Client Connect}
 \item \href{https://pusher.com/docs/client_api_guide/client_events}{Client Events}
\end{enumerate}

% Ongoing development
\section{Further Development}
Currently, the API is still in improvements and many new other aspects of truth finding and endpoints are still under investigation, based on recent
proposed development in the field, discussions with early adopter developers, and users' feedbacks. For example, the scheme may support HTTPS in the 
future, the response body format may support other output formatting like XML.Most importantly, the API must be extended in order to propose broader 
class of truth finding algorithms, in particular the new ones.
in the future.
% AllegatorTrack application
\section{Online Tool}
The main goal of DAFNA project is to design a scalable and accurate truth discovery system to score the veracity of information extracted from multiple 
online sources.  We design DAFNA API as the first proposed API that allows users to tests and compares a large class of well-known truth finding algorithms
on different scenarios and datasets. AllegatorTrack is the first application that uses DAFNA API and provides to users
an online plateform for testing the different features of the API. AllegatorTrack application is accessible via \href{http://dafna.qcri.org/users/sign\_in}{http://dafna.qcri.org/users/sign\_in}.
\end{document}
