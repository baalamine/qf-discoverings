% Information Extraction Technique
\section{Open Information Extraction}
\laure{
\begin{itemize}
 \item décrire le type d'information auquel on s'intéresse par exemple "factoid claim"
 \item decrire le systeme sur lequel on se base
 \item décrire comment on transforme l'output de OpenIE
 \item donner qq exemples
\end{itemize}
}

In this study, we are interested on truth discovering on the large number of ``factoid" statements
(or claims) made by multiple Web sources in the same real-world domain. A ``factoid" claim is
a piece of unverified or inaccurate information that is presented as factual, often
as part of a publicity effort. Such type of claims  is usually accepted as true  because of its
frequent redundancy over multiple sources. We focus on conflicting factoid claims provided by typical 
open Web information extraction systems as answers to users' queries. Concretly, given user input, 
we retrieve the set of candidates claims, together with the associated sources, returned by TextRunner~\footnote{TextRunner is an online
at \href{http://openie.allenai.org/}{http://openie.allenai.org/}}
from a Web corpus. We then format this output in such a way that fits our truth discovering process.



% OIE Input and Ouput
\paragraph*{TextRunner System}
TextRunner relies on an unsupervised extraction procedure which queries, using a single and data-driven 
pass, an entire Web corpus of unstructured texts and extracts a list of candidate relational tuples with 
respect to user input. Such a user input query consists typically of a sentence formed by
two \emph{entities} and a certain \emph{relation}. Its semantics being that the user is looking for a set 
of claims on the Web that support or not the relation specified between the two entities. Formally, a user
query in TextRunner can be defined as a triplet $(\e{1}, \rel{1,2}, \e{2})$ where $\e{1}$ and $\e{2}$ correspond 
to the entities of the query and $\rel{1,2}$ represents the specified relationship between them. Note that the 
extactor accepts an \emph{incomplete} query, that is, a query that does not contain all the three components. In 
real scenarios, a given user is often searching for claims about incomplete relations. 
 
Given this input, the extractor searches into a corpus, consisting
of several thousand of sentences, the collections of relational tuples that might satisfy the searched relation.
In other terms, the extractor outputs a set of tuples $<(e_i, r_{i,j}, e_j)>$.

\paragraph*{Formatting of Data Sources}
We use TextRunner for the extraction of the set of 
candidates claims that might correspond to the user query.