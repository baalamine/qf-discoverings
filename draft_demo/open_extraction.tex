% Information Extraction Technique
\selectlanguage{english}
\section{Open Information Extraction}
\begin{itemize}
 \item décrire le type d'information auquel on s'intéresse par exemple "factoid claim"
 \item decrire le systeme sur lequel on se base
 \item décrire comment on transforme l'output de OpenIE
 \item donner qq exemples
\end{itemize}


In this study, we are interested on truth discovering on the large number of ``factoid" statements
(or claims) made by multiple Web sources in the same real-world domain. A ``factoid" claim , e.g., 
\emph{Barack Obama was born in Kenya}, is
a piece of unverified or inaccurate information that is presented as factual, often
as part of a publicity effort. Such type of claims  is usually accepted as true  because of its
frequent redundancy over multiple sources. We focus on conflicting factoid claims provided by typical 
open Web information extraction systems as answers to users' queries. Concretly, given user input, 
we retrieve the set of candidates claims, together with the associated sources, returned by TextRunner~\footnote{TextRunner is an online
at \href{http://openie.allenai.org/}{http://openie.allenai.org/}}
from a Web corpus. We then format this output in such a way that fits our truth discovering process.



% OIE Input and Ouput
\paragraph*{TextRunner extractor}
The extraction engine relies on an unsupervised extraction procedure which queries, using a single and data-driven 
pass, an entire Web corpus of unstructured texts and extracts a list of candidate relational tuples which might
satisfy a given user input query. A user query, in this setting, consists typically of a sentence formed by three
components: two real-world \emph{entities} and a certain \emph{relation}. The goal of TextRunner being to find and
gather the set of possible claims on the Web which support or not the specified relation between the two entities. 
A user query $\query$ can be defined formally with the help of a triplet $(\e{1}, \rel{}, \e{2})$ where $\e{1}$ and $\e{2}$ 
are real-world entities and $\rel{}$ is a relation; $\rel{}$ indicates a possible relationship between the
two given entities. If some components, i.e., the second entity, of the query are not provided by the user, we 
say that the query is \emph{incomplete}. Incomplete queries are common in Web search as human often has a partial
knowledge of the real world: this motivates us the use of information extractors in order to complete his knowledge.
Note that the information extraction systems do also well in the case of \emph{incomplete} queries.

The output of TextRunner is a set of candidate claims which are ranked according to their number of sources.A claim in this
context can correspond to a tuple, a relation, or an real-world entity.
The system also offer the possibility, through pointers, to access to these sources and the full corpus in which the each
has been extracted. We would harness these pointers in order to format the result obtained from the extraction system  in
such the way that it fits the input of our differents truth finding algorithms. We show in the following how such a 
formatting is realized.

\paragraph*{Data formatting}
Let given a user input tuple $\query=(\e{1}, \rel{},\e{2})$ on TextRunner. We refer to the set 
of $n$ claims returned by the system as $\claim{1}\ldots \claim{n}$ as candidate answers the query
$\query$. We suppose that the query admits only one true answers, thereby we are in the presence of
$n$ conflicting answers. For each returned claim, we go through the linked pointer and extract the
corresponding source by using hand-written mapping rule. We denote the set of $m$ sources queried by 
the extraction system in order to answer $\query$ by $\source{1}, \ldots, \source{m}$.