% Information Extraction Technique
\section{Open Information Extraction}
\laure{
\begin{itemize}
 \item décrire le type d'information auquel on s'intéresse par exemple "factoid claim"
 \item decrire le systeme sur lequel on se base
 \item décrire comment on transforme l'output de OpenIE
 \item donner qq exemples
\end{itemize}
}

In this study, we are interested on truth discovering over a set of ``factoid" statements
(or claims) made by multiple Web sources in the same real-world domain. A ``factoid" claim is
a piece of unverified or inaccurate information that is presented as factual, often
as part of a publicity effort. Such type of claims  is usually accepted as true  because of its
frequent redundancy over multiple sources. We focus on conflicting factoid claims provided by open
Web information extraction systems as answers to users' queries. Specifically, given a user's input, 
we retrieve the set of candidates claims, together with the associated sources, returned by TextRunner
from a Web corpus. 



% OIE Input and Ouput
\paragraph*{TextRunner Input and Output}
TextRunner system performs an unsupervised extraction by requying, in a single and data-driven manner,
an entire Web corpus of unstructured texts and by extracting a list of relational tuples without 
any human intervention.

The input $t$ of the extraction system is a triplet $(e_i, r_{i,j}, e_j)$ where $e_i$ and $e_j$ are called
\emph{entities} and $r_{i,j}$ is a \emph{relation}. The semantics of $r_{i,j}$ is that that of a certain 
relationship between the two given entities. Given this input, the extractor searches into a corpus, consisting
of several thousand of sentences, the collections of relational tuples that might satisfy the searched relation.
In other terms, the extractor outputs a set of tuples $<(e_i, r_{i,j}, e_j)>$.

\paragraph*{Source and Claim Formatting}
We use TextRunner for the extraction of the set of 
candidates claims that might correspond to the user query.