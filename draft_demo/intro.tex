\section{Introduction}
%\lamine{Page allocation}
%\begin{itemize}
 %\item 1.25 pages --> abstract + introduction
 %\item 1.5 pages --> Open information extraction + Active Ensembling for Truth Discovery 
 %\item 1 pages --> Demonstration System + Scenario
 %\item 0.25 pages --> References
%\end{itemize}

%\paragraph{Use cases}
%\begin{itemize}
% \item \textbf{Information extraction improvement:} Truth discovery over claims returned by OpenIE systems, e.g., TextRunner
% \item \textbf{Online hot news verification:} truth discovery over factual claims in each new's headline and
% content published on the online front page of AlJazeera
%\end{itemize}

Online fact-checkers such as FactCheck\footnote{FactCheck, {\small\url{ http://www.factcheck.org/}}}, Snopes\footnote{Snopes, {\small\url{ http://www.snopes.com/}}}, PolitiFact\footnote{PolitiFact,{\small \url{ http://www.politifact.com/}}}, TruthorFiction\footnote{TruthorFiction, {\small\url{ http://www.truthorfiction.com/}}} or OpenSecrets\footnote{OpenSecrets, {\small \url{ http://www.opensecrets.org/}}}) and ClaimBuster\footnote{ClaimBuster, {\small\url{ http://idir-server2.uta.edu/claimbuster}}} 
 have recently gained  unprecedented attention as their legitimate goal is to verify online information for enlightening public opinion and automate Web-scale fact-checking for assisting journalists \citep{Cohen2011,Hassan:2015}.
 However, estimating the veracity of data remains a challenging problem: extracting information from large, heterogeneous corpora of textual and multimedia documents and  integrating the extracted, multi-source data are difficult tasks. Data can be noisy, outdated, incorrect, conflicting, and thus unreliable, mainly due to information extraction errors,  low source quality and disagreements between the information sources. 
 
 \laure{[ ici: point sur etat de l'art des methods de truth discovery et nécessité de faire ensembling car one-fits-all solution does not exist for open  domain + nécessité d'intégrer/developper full truth discovery pipeline incluant l'extraction]}
 
In this demo, we present {\scschape Vera}, a web-platform that supports information extraction, data fusion, truth finding, and visualization of large-scale, multisource data from the Web.

The main contributions of our work are:

  \laure{[ici: lister contrib]}
 
 
{\scschape Vera} platform, REST API and additional information, data sets and material are available at \url{dafna.qcri.org}.


 
