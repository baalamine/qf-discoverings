\documentclass{sig-alternate}
\usepackage[utf8]{inputenc}
\usepackage[english, francais]{babel}
\usepackage{microtype}
\usepackage{xspace}
\usepackage{amsfonts}
\usepackage{amssymb}
\usepackage{mathtools}
\usepackage{xcolor}
\usepackage{paralist}
\usepackage{booktabs}
\usepackage{wrapfig}
\usepackage{subfigure}
\usepackage{csquotes}
\usepackage{tikz}
\usepackage[pdfborder={0 0 0}, pdfpagelabels=false]{hyperref}
\usetikzlibrary{matrix,shapes.multipart, arrows, shapes.geometric,fit,scopes}
\tikzset{
  >= latex,
  el/.style={ellipse, draw, text width=8em, align=center},
  rs/.style={rectangle split, draw, rectangle split parts=#1},
  ou/.style={draw, inner xsep=1em, inner ysep=1ex, fit=#1}
}
\definecolor{marron}{rgb}{0.8,0.6,0.4}
\newcommand{\lamine}[1]{\textbf{[Lamine: {\textcolor{red}{#1}}]}}
\newcommand{\laure}[1]{\textbf{[Laure: \textcolor{marron}{#1}]}}
\newcommand{\tbd}[1]{\textbf{[TODO: \textcolor{blue}{#1}]}}
\newcommand{\qset}{\ensuremath{\mathcal{Q}}}
\newcommand{\query}{\ensuremath{\mathit{q}}}
\newcommand{\fact}[1]{\ensuremath{\mathit{f}_{#1}}}
\newcommand{\e}[1]{\ensuremath{\mathit{e_{#1}}}}
\newcommand{\claim}[1]{\ensuremath{\mathit{c_{#1}}}}
\newcommand{\claimset}{\ensuremath{\mathcal{C}}}
\newcommand{\rel}[1]{\ensuremath{\mathit{r}_{#1}}}
\newcommand{\source}[1]{\ensuremath{\mathit{s}_{#1}}}
\newcommand{\sset}[1]{\ensuremath{\mathcal{S}_{#1}}}
\newcommand{\TRUE}{\ensuremath{\mathsf{true}}}
\newcommand{\FALSE}{\ensuremath{\mathsf{false}}}


\begin{document}
\conferenceinfo{}{}

\title{{\scshape Vera}: A Platform for Estimating the Veracity of Web Information}

\numberofauthors{1} 
\author{
\alignauthor
Mouhamadou Lamine Ba, Laure Berti-Equille, Hossam M. Hammady, Yasser Idris\\
       \affaddr{Qatar Computing Research Institute}\\
       \affaddr{Tornado Tower, West Bay}\\
       \affaddr{Doha, Qatar}\\
       \email{\{mlba,lberti,hhammady,yidris\}@qf.org.qa}
% 2nd. author
%\alignauthor
%\\
%       \affaddr{Qatar Computing Research Institute}\\
%       \affaddr{Tornado Tower, West Bay}\\
%       \affaddr{Doha; Qatar}\\
 %      \email{lberti.qf.org.qa}
%\and  % use '\and' if you need 'another row' of author names
}


\maketitle

% Page allocation for this demo
% 1.25 pages --> abstract + introduction
% 1.5 pages --> Active Truth finding Process
% 1 pages --> Demonstration system
% 0.25 pages --> references
\begin{abstract}
%Over the last decade there's been a substantial growth in the use of fact-checking and truth discovery techniques to detect misinformation available online. 
Multiple Web information sources often claim conflicting data and estimating data veracity is extremely difficult especially when no prior knowledge about the sources or the claims is available. However, exploring the space of conflicting data and the polarity of Web sources claiming them is relevant in this context.
This demo presents Vera, a Web-based platform that supports event, entity and relation
extraction from Web information sources, systematically processes the extracted, conflicting claims, and
combines multiple truth finding algorithms with active learning to return data veracity and controversy scores. Vera also determines the most trustworthy sources. Vera will be demonstrated throught several real-world use cases.
\end{abstract}

% A category with the (minimum) three required fields
%\category{H.4}{Information Systems Applications}{Miscellaneous}
%A category including the fourth, optional field follows...
%\category{D.2.8}{Software Engineering}{Metrics}[complexity measures, performance measures]

%\terms{Theory}

%\keywords{ACM proceedings, \LaTeX, text tagging}

% Introduction
\section{Introduction}
\lamine{Proposition of page allocation for the demo paper}
\begin{itemize}
 \item 1.25 pages --> abstract + introduction
 \item 1.5 pages --> Active Ensembling for Truth Discovery for Open Information
 \item 1 pages --> Demonstration System + Scenario
 \item 0.25 pages --> References
\end{itemize}


% Open Information Extraction
% Information Extraction Technique
\selectlanguage{english}
\section{Open Information Extraction}
\begin{itemize}
 \item décrire le type d'information auquel on s'intéresse par exemple "factoid claim"
 \item decrire le systeme sur lequel on se base
 \item décrire comment on transforme l'output de OpenIE
 \item donner qq exemples
\end{itemize}


In this study, we are interested on truth discovering on the large number of ``factoid" statements
(or claims) made by multiple Web sources in the same real-world domain. A ``factoid" claim , e.g., 
\emph{Barack Obama was born in Kenya}, is
a piece of unverified or inaccurate information that is presented as factual, often
as part of a publicity effort. Such type of claims  is usually accepted as true  because of its
frequent redundancy over multiple sources. We focus on conflicting factoid claims provided by typical 
open Web information extraction systems as answers to users' queries. Concretly, given user input, 
we retrieve the set of candidates claims, together with the associated sources, returned by TextRunner~\footnote{TextRunner is an online
at \href{http://openie.allenai.org/}{http://openie.allenai.org/}}
from a Web corpus. We then format this output in such a way that fits our truth discovering process.



% OIE Input and Ouput
\paragraph*{TextRunner extractor}
The extraction engine relies on an unsupervised extraction procedure which queries, using a single and data-driven 
pass, an entire Web corpus of unstructured texts and extracts a list of candidate relational tuples which might
satisfy a given user input query. A user query, in this setting, consists typically of a sentence formed by three
components: two real-world \emph{entities} and a certain \emph{relation}. The goal of TextRunner being to find and
gather the set of possible claims on the Web which support or not the specified relation between the two entities. 
A user query $\query$ can be defined formally with the help of a triplet $(\e{1}, \rel{}, \e{2})$ where $\e{1}$ and $\e{2}$ 
are real-world entities and $\rel{}$ is a relation; $\rel{}$ indicates a possible relationship between the
two given entities. If some components, i.e., the second entity, of the query are not provided by the user, we 
say that the query is \emph{incomplete}. Incomplete queries are common in Web search as human often has a partial
knowledge of the real world: this motivates us the use of information extractors in order to complete his knowledge.
Note that the information extraction systems do also well in the case of \emph{incomplete} queries.

The output of TextRunner is a set of candidate claims which are ranked according to their number of sources.A claim in this
context can correspond to a tuple, a relation, or an real-world entity.
The system also offer the possibility, through pointers, to access to these sources and the full corpus in which the each
has been extracted. We would harness these pointers in order to format the result obtained from the extraction system  in
such the way that it fits the input of our differents truth finding algorithms. We show in the following how such a 
formatting is realized.

\paragraph*{Data formatting}
Given a user input tuple $\query=(\e{1}, \rel{},\e{2})$, we refer to the
set of $n$ claims $\claim{1}\ldots \claim{n}$ extracted and returned by the
extration engine as potential answers. 
We assume that the query admits only one true answers, thereby we are in the presence of
$n$ conflicting answers. For each returned claim, we go through the linked pointer and extract the
corresponding source by using hand-written mapping rule. We denote the set of $m$ sources queried by 
the extraction system in order to answer $\query$ by $\source{1}, \ldots, \source{m}$.

% Active ensembling for Truth Finding
\section{Active Truth Finding Process}

% Information Extraction Technique
\subsection{Open Information Extraction}

% Truth discoverying over multiple answers
\subsection{AllegatorTrack Application}

% Active Truth Finding with User Feedbacks
\subsection{Online Learning Process}


% Demonstration system
\section{Our Demonstration System}

% Architecture and GUI
\subsection{System Architecture and User Interface}
The architecture of our demonstration system, given in
Figure~\ref{system_architecture}, comprises the following
three main components.

% User I/O Interface
\paragraph*{User I/O Interface}It represents the main entry point
of our application for user interaction. The user I/O interface is
composed by a text search area where a given user can enter its 
search keywords, in terms of a relation, The final result of the 
overall process will be also show to the users through this component.
Finally, the user gives it feebacks via the user I/O interface through
the button options or the form.

% Information extraction module
\paragraph*{Information extraction module} This is the information 
extraction module which considers the input of the user and browsers
several Web sources in order to returns the relevant answers. In our 
system, we rely on TextRunner in order to extract information from Web corpus.

% Truth finding engine
\paragraph*{Truth Finding Engine} It corresponds to AllegatorTrack which contains
twelve truth finding algorithms with different accuracy according to the types of 
claims and the characteristics of sources.

% Learning module
\paragraph*{Learning Module} We have also a learning method that uses our knwoledge
bases of users feedbacks. It enables to learn about the best truth finding algorithms,
among the twelve, to use with respect to the type of entities or relations searched by 
the user.

% knwoledge bases
\paragraph*{Repository of Labeled Facts}The knowledge base contains the information used for the learning
phase the truth finding procedure. These information include the true facts for some relations
which have been learnt based on the feedbacks of the users. In addition, our knowledge base could
be enriched with ground truth about some facts from reliable sources such as Wikipedia. Based on 
the knowledge base, our system has the ability to improve the accuracy of the truth finding process
by learning about the best method to use or the best parameters, e.g., sources' accuracy scores, to 
consider for a better boostrapping of the process.

\begin{figure}[ht]
\begin{tikzpicture}[auto, line/.style ={draw,  thick, shorten <=0pt}]
  \matrix[matrix of nodes, row sep=5ex, column sep=1em] (mx) {
    User I/O Interface &\\
    IE Module &  Truth Finding Module\\
    & Learning Module\\ 
    & Repository of Labeled Facts\\
  };
  %\node[ou=(mx-1-3)] (empty) {};
  \node[ou=(mx-2-1)] (oi) {};
  \node[ou=(mx-2-2)] (tf) {};
  \node[ou=(mx-3-2)] (lm) {};
  \node[ou=(mx-4-2)] (kb) {};
  %{[->, thick]
  %\draw(mx-1-1)edge(empty);
  %}
  {[->, thick]
  \draw(mx-1-1)edge(oi);
  \draw(oi)edge(tf);
  }
  {[<->]
    \draw(tf)edge(lm);
     \draw(lm)edge(kb);
  }
  \begin{scope}[every path/.style=line, <-]
   \path(mx-1-1)  - | (tf);
  % \path(mx-1-3)  - | (lm);
  \end{scope}

\end{tikzpicture}
\caption{Architecture of our system}\label{system_architecture}
\end{figure}
% Demonstration scenario
\subsection{Demonstration Scenario}
A given user that wants to interact with our system
must do it through the search form. Through the search
form, she (or he) provides her searched relation, e.g.,
``Where is born Barack Obama?".
The searched relation is then 
passed to the information extraction engine,  TextRunner system
in our case, which returns a set of answers considered to be 
relevant for the user's request. Each claim in the returned list is
processed in order to extract the corresponding sources along a detailed
description of the claim which we format in a certain manner. The 
set of sources and the formatted versions of all claims are then passed
to the truth finding module which integrate all the claims and compute the
most probable answer together with the reliability scores of participated 
sources for the searched relation. Finally, the output of the truth finding
process is returned to the user. The user can also want to review the output
of our system by definitively validiting it or not through its knwoledge of the
modeled world. For example when the system has totally wrong, it may be interesting
to get such a kind of feedbacks from the user in order to change the used method, as
there are many available with our system, and to enhance the process for the further 
search about the same world. The user gives feedbacks using the option buttons on the 
left-hand side of the outputted claims or the text form. The feebacks given by the user
is saved in knwoledge bases within our system for further processes.


% Conclusion
\section{Conclusion}
%\lamine{L'utilisateur peut faire une erreur sur l'\'etiquette de certains claims. Comment capturer ce ph\'enom\'ene?}
% References
%\cite{*}
\bibliographystyle{plain}
\bibliography{biblio}

\end{document}
