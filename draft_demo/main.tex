\documentclass{sig-alternate}
\begin{document}
\conferenceinfo{}{}

\title{Active Truth Finding for relevant Information Retrieval}

\numberofauthors{2} 
\author{
\alignauthor
Mouhamadou Lamine BA\\
       \affaddr{Qatar Computing Research Institute}\\
       \affaddr{Tornado Tower, West Bay}\\
       \affaddr{Doha, Qatar}\\
       \email{mlba@qf.org.qa}
% 2nd. author
\alignauthor
Laure Berti-Equille\\
       \affaddr{Qatar Computing Research Institute}\\
       \affaddr{Tornado Tower, West Bay}\\
       \affaddr{Doha; Qatar}\\
       \email{lberti.qf.org.qa}
%\and  % use '\and' if you need 'another row' of author names
}


\maketitle
\begin{abstract}
Open Web Information extraction systems like TextRunner~\cite{Yates07}
or popular Web search engines such as Google or Bind usually reply to
users' search queries by returning a set of potential relevant answers. Given
some specific type of queries, the returned list might contain conflicting answers
which make thing harder for the end-users to distinguish between the truth and 
the false.
We demonstrate in the paper a system that processes answers outputted 
by open Web information extraction systems like TextRunner and provides 
the most probable answer using truth finding. Our system has also the 
capability to account for users' feedbacks, based on its knowledge of the 
correct instances for some searched relations, in order to improve the truth
finding process.
\end{abstract}

% A category with the (minimum) three required fields
%\category{H.4}{Information Systems Applications}{Miscellaneous}
%A category including the fourth, optional field follows...
%\category{D.2.8}{Software Engineering}{Metrics}[complexity measures, performance measures]

%\terms{Theory}

%\keywords{ACM proceedings, \LaTeX, text tagging}

% Introduction
\section{Introduction}

% Active Truth Finding for relevant information search
\section{Active Truth Finding}

% Information Extraction Technique
\subsection{Information Extraction Technique}

% Truth discoverying over multiple answers
\subsection{AllegatorTrack Module}

% Active Truth Finding with User Feedbacks
\subsection{Active Truth Finding with User Feedbacks}

% Our demonstration application
\section{Our Demonstration System}

% Architecture and GUI
\subsection{Architecture and GUI}

% Demonstration scenario
\subsection{Demonstration Scenario}

% Conclusion
\section{Conclusion}

% References
\bibliographystyle{plain}
\bibliography{biblio}

\end{document}
