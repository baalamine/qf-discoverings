% Active Ensembling for Truth Finding
\section{Active Ensembling for Truth Finding}

\subsection{Ensembling }
- donner idée générale pour introduire  ce qu'est l'ensembling
- on a besoin de le faire dans le contexte de truth discovery car aucune methode ne bat toutes les autres dans tous les cas de figure
- donc on combine les methodes : il y plusieurs façon de combiner par ex. consensus de méthodes, etc.
- expliquer quelles méthodes on combine avec leurs avantages et inconvénients

An ensemble is a supervised learning algorithm in the sense that is can be trained and then used to make predictions.
An ensembling, or  commonly an ensemble-based active learning, is a learning process selecting one classifier type, or appropriate combinaisons 
of multiple classifier types, to construct ensembles for a given tasks.

\subsection{Active Learning Process}
- notre approche que l'on défend ici dans la démo est  semi supervisée en impliquant de l'utiliseur de façon active
en lui demandant s'il peut confirmer des faits (facts)
- si on a une ground truth partielle on la "rejoue" cas par cas

