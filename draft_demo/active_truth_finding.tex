% Active Ensembling for Truth Finding
\section{Active Ensembling for Truth Finding}
We describe in this section our use of active ensembling in the context of truth finding
for discovering gradually the optimal set of algorithms that together maximizes the accuracy
of the process when users' feedbacks arrive into the system.

\subsection{Ensembling }
\begin{itemize}
\item donner idée générale pour introduire  ce qu'est l'ensembling
\item on a besoin de le faire dans le contexte de truth discovery car aucune methode ne bat toutes les autres dans tous les cas de figure
\item donc on combine les methodes : il y plusieurs façon de combiner par ex. consensus de méthodes, etc.
\item expliquer quelles méthodes on combine avec leurs avantages et inconvénients
\end{itemize}


Ensembling, also known as ensemble-based active learning, is a supervised learning approach that learns
about an appropriate combinaison of multiple classifier types for a given task. In our setting, we are interesting
on discovering the best combinaison among several possible truth finding algorithms. This combinaison is excepted 
to maximize the precision of truth finding process over a given dataset characteristics.
One well known lack, e.g., see in~\cite{Li12, Wagui14}, of existing truth discovering algorithms is the fact that
they are mostly sensitive to given application domains and data characteristics. As a consequence, there is no approach 
that outperforms the others on all types of dataset.  An ensembling technique might enable  us to combine multiple algorithm
in order to obtain an optimal hybrid truth finding approach that outperform any individual approach. Furthermore, the truth
finding process become more tricky in practical scenarios in the sense that there usually exists no \emph{ground truth data}
against which one can evaluate the precision of the different algorithms. However, users have often some knowledge background
of the truth about certain real world facts. These valuable feebacks from the user could serve as partial gold standard in order
to estimate the accuracy of each truth finding algorithm.  As we shall show in later, we actively query users for labeled data to 
use in our active approach.

Several query strategies, e.g., uncertainty sampling or query by committee, could be adopted in order to obtain from users labeled
examples for the active learning process; see~\cite{burr12} for more details about active learning. In this study, we have used query 
by committee approach which is proven to be very effective in many different settings.



Our ensemble-based active learning is performed by considering twelve well established truth finding algorithms which range from naive approaches
to sophisticated ones. We briefly introduce each considered class of truth disc in the following.

\begin{enumerate}
 \item Naive techniques: The class is essentially formed by \emph{majority voting} which considers the truth as providing the majority of 
sources. This class assumes that the sources are equally reliable.
 \item Iterative techniques
 \item EM based techniques
 \item Dependency detection based techniques
\end{enumerate}



\subsection{Active Learning Strategy}
\begin{itemize}
 \item notre approche que l'on défend ici dans la démo est  semi supervisée en impliquant de l'utiliseur de façon active
en lui demandant s'il peut confirmer des faits (facts)
\item si on a une ground truth partielle on la "rejoue" cas par cas
\end{itemize}

