% Information Extraction Technique
\subsection{Open Information Extraction}
-  décrire le type d'information auquel on s'intéresse par exemple "factoid claim"
- decrire le systeme sur lequel on se base
- décrire comment on transforme l'output de OpenIE
- donner qq exemples

\section{Active Ensembling for Truth Finding}

\subsection{Ensembling }
- donner idée générale pour introduire  ce qu'est l'ensembling
- on a besoin de le faire dans le contexte de truth discovery car aucune methode ne bat toutes les autres dans tous les cas de figure
- donc on combine les methodes : il y plusieurs façon de combiner par ex. consensus de méthodes, etc.
- expliquer quelles méthodes on combine avec leurs avantages et inconvénients

\subsection{Active Learning }
- notre approche que l'on défend ici dans la démo est  semi supervisée en impliquant de l'utiliseur de façon active
en lui demandant s'il peut confirmer des faits (facts)
- si on a une ground truth partielle on la "rejoue" cas par cas

% Truth discoverying over multiple answers
%\subsection{AllegatorTrack Application}

% Active Truth Finding with User Feedbacks

