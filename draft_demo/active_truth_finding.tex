% Active Ensembling for Truth Finding
\section{Active Ensembling for Truth Finding}
We describe in this section our use of active ensembling in the context of truth finding
for discovering gradually the optimal set of algorithms that together maximizes the accuracy
of the process when users' feedbacks arrive into the system.

\subsection{Ensembling }
\begin{itemize}
\item donner idée générale pour introduire  ce qu'est l'ensembling
\item on a besoin de le faire dans le contexte de truth discovery car aucune methode ne bat toutes les autres dans tous les cas de figure
\item donc on combine les methodes : il y plusieurs façon de combiner par ex. consensus de méthodes, etc.
\item expliquer quelles méthodes on combine avec leurs avantages et inconvénients
\end{itemize}


Ensembling, or commonly an ensemble-based active learning, is a supervised learning approach that learns
about an appropriate combinaison of multiple classifier types for a given task. In our setting, we are interesting
on discovering the best combinaison among several possible truth finding algorithms. This combinaison is excepted 
to maximize the precision of truth finding process over a given dataset characteristics. An ensembling technique 
might enable to find a more accurate hybrid truth finding approach that will outperform any individual algorithm.
One well known lack, e.g., see in~\cite{Li12, Wagui14}, of existing truth discovering algorithms is the fact that
they are mostly sensitive to given application domains and data characteristics.



We learn on various classes of truth finding techniques, ranging from naive classes to more elaborated ones. We
briefly detail each class in the following.
\begin{enumerate}
 \item Naive techniques
 \item Iterative techniques
 \item EM based techniques
 \item Dependency detection based techniques
\end{enumerate}



\subsection{Active Learning Process}
\begin{itemize}
 \item notre approche que l'on défend ici dans la démo est  semi supervisée en impliquant de l'utiliseur de façon active
en lui demandant s'il peut confirmer des faits (facts)
\item si on a une ground truth partielle on la "rejoue" cas par cas
\end{itemize}

