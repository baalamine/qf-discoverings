% Active Ensembling for Truth Finding
\section{Ensembling for Truth Finding}\label{ensembling}
We present in this section an adaptive truth finding approach which uses active ensembling in order to 
adaptively learn about an optimal set of truth finding algorithms that outperforms any individual
technique on any given dataset. Our learning approach will actively involve users for the correct
labels (or answers) of a sample of queries that cause maximal disagreement amongst our classifiers.

\subsection{Ensemble-Based Active Learning}
\begin{itemize}
\item donner idée générale pour introduire  ce qu'est l'ensembling
\item on a besoin de le faire dans le contexte de truth discovery car aucune methode ne bat toutes les autres dans tous les cas de figure
\item donc on combine les methodes : il y plusieurs façon de combiner par ex. consensus de méthodes, etc.
\item expliquer quelles méthodes on combine avec leurs avantages et inconvénients
\end{itemize}

\medskip
An ensemble-based active learning, or commonly ensembling, is a semi-supervised learning approach that tries
to figure out an optimal ensemble of classifiers for a given classification problem by actively querying an 
oracle, e.g., a human being, about the labels of a sample of data items. Ensembling, thereby, enables to perform
classification consistently well across datasets without having to determine a \emph{priori} a suitable classifier 
type.

In our context, the truth finding algorithms correspond to our set of classifiers. The underlying \emph{binary classification} 
problem consists of assigning the correct truth label to a set of claims about given user queries. Indeed, a truth finding 
algorithm is formally a mapping $\textsf{TF}:~\claimset \mapsto \{\TRUE, \FALSE\}$ which associates to each claim in $\claimset$
either $\TRUE$ or $\FALSE$. A good truth finding algorithm provides  predictions that are close to the actual world. Unfortunately, 
a well known property, e.g., as shown in~\cite{Li12, Wagui14}, of existing truth discovering algorithms remains their sentivity
to certain application domains or datasets. As a consequence, there is no actual approach that outperforms the others on all types 
of datasets. On the other hand, truth finding is hard in practical scenarios because there is often no prior knowledge guiding to
the selection, beforehand, of an optimal algorithm, in particular when the context is dynamic. More importantly, a large set of labeled 
examples (or ground truth) for evaluating the precisions of the algorithms is expensive to obtain in real applications. 

In general, human being has a certain background knowledge about some real-world facts. Such a knowledge can serve as a valuable and inexpensive source of labels for a 
rather reasonable number of data items. However, having this partial ground truth from users is not sufficient in order to definitively decide about an optimal truth
finding strategy because it can change over time as we obtain more information from sources, e.g. when claims are continuously extracted by TextRunner for answering
new incoming queries. Therefore, there is a need for an adaptive approach able to dynamically figure out the optimal truth finding strategy when users' feedbacks and 
new knowledge about the world are available. We believe that active ensembling should be helpful to this end.

We put forward and demonstrate an approach which combines truth discovery and open information extraction with ensemble-based active learning for adaptively learning about the optimal 
ensemble of truth finding algorithms when the OpenIE system is gradually querying and labeled examples from users are available.
As we shall show later, we will actively involve users to obtain the truth about a sample of particular facts during the learning process. The way this sampling is defined and
chosen is crucial for the effectiveness of the active learning. Several sample selection strategies, e.g., random sampling, query by committee, or support vector machine models,
have been proposed for the definition of the type of selected data items along the size of the sample; we defer to~\cite{burr12} for more details about active machine learning.
In this study, we use \emph{query by committee} (QBC) for ensemble-based active learning. QBC states that the best data items to select for labels are those that cause 
the \emph{maximal disagreement} among the predictions of an ensemble of diverse but partially accurate classifiers during active learning. Furthermore, we seek to provide
an adaptive active learning by looking for an optimal ensemble given a larger set of input classifiers.
\lamine{Peut \^etre qu'il y a mieux que QBC ?}

To learn about an optimal ensemble from a diverse set of classifiers, we have considered
twelve well established truth finding algorithms in the literature, having three different types 
according to their specificities. Note that diversity offers better result in active learning than
using homogenoeus classifiers (see~\cite{Lu15}). We briefly present each considered class of truth
discovering algorithms in the following.

\begin{enumerate}
 \item \textbf{Iterative techniques:} TruthFinder~\cite{YinHY08}, Cosine, 2-Estimates and 3-Estimates~\cite{GallandAMS10}, 
 AccuNoDep~\cite{DongBS09}
 \item \textbf{EM based techniques:} MLE~\cite{WangKLA12}, LTM~\cite{ZhaoRGH12}, SimpleLCA and GuessLCA~\cite{PasternackR13}
 \item \textbf{Dependency detection based techniques:} Depen, Accu, and AccuSim~\cite{DongBS09}
\end{enumerate}

\lamine{Peut \^etre qu'il existe une meilleure classification ?}


\subsection{Truth Finding with Active Ensembling}
\begin{itemize}
 \item notre approche que l'on défend ici dans la démo est  semi supervisée en impliquant de l'utilisateur de façon active
en lui demandant s'il peut confirmer des faits (facts)
\item si on a une ground truth partielle on la "rejoue" cas par cas
\end{itemize}

\medskip

We present an adaptive truth finding algorithm based on active ensembling in order to learn about an optimal
ensemble over a set of existing truth finding algorithms, on which one can efficiently find the truth for the output 
of OpenIE systems. Our approach first obtains from the learning procedure intuitions about the best algorithm to use 
for each incoming fact (or a query about it) and then it performs the union of the result of the ensemble of best truth finding
algorithms returned for a collection of facts. The best truth finding algorithm for a fact, i.e., the algorithm that has the highest
chance to reliability discover the truth among a set of candidates claims about this fact, will be found by the learner by evaluating
the accuracy of each competing technique on labeled claims from the user. We sketch in the following the procedure by assuming that all
the input truth finding algorithms are used with their optimal initial parameters which are deemed known beforehand.

\lamine{A sketch of the active learning process for truth finding}
Our ensemble-based learning process relies on QBC for label querying and aims at finding an optimal combinaison of the result
of the different truth finding algorithm involved in the process. Such a learning process is iterative. Consider the set of 
all claims $\claimset{}$ and the set of labeled claims $\claimset_{GT}$. The claims in $\claimset{}$ are progressively obtained
and processed, like in a streaming manner, as from the information extraction system as long as it receives queries.
The set $\claimset_{GT}$, initially empty, contains
claims whose labels are known for sure by asking the user. We also assume a stop criteria (e.g., a predefined number of iterations 
or accuracy changes between two iterations) for our iterative learning algorithm for truth finding and the set of optimal initial
parameters for each truth finding algorithm.
The algorithm starts by evaluating the truth finding algorithms on the available unlabeled claims in $\claimset{}$ for determining
the prediction of each technique. It then compare the set of the different algorithms on each set of claims about the same query (or fact)
and determines the queries causing the maximal disagreement among the members of the committee. Those queries are determined by computing 
the vote entropy of each query. The algorithm requests to the user labels for the set of candidates claims of those queries. Once the labels
are acquired from the user, the algorithm adds those labels along the associated claims into the labeled set $\claimset_{GT}$ and discards the
claims from the unlabeled set $\claimset{}$. The learning procedure finally determinates the accuracy of each input truth finding algorithm on
the newly labeled claims in order to know the best technique for truth discovery. 

\begin{enumerate}
\item The algorithm starts with an initialization phase in which values the initial parameters of the learner and the different truth 
finding algorithm are set.
 \item The algorithm pursues by giving the claims in $\claimset{}$ to the set of truth finding algorithms for label predictions and then 
 it records the vote entropy of each query (thereby the underlying fact) according to the predictions of the committee.
 \item It chooses the set of claims associated to the query having the maximum vote entropy for label querying
 \item The set of acquired truth labels, together with the corresponding claims, are added into  $\claimset_{GT}$ and then discarded from $\claimset{}$.
 \item At this stage the algorithm estimates the accuracy of the different truth finding algorithm on  $\claimset_{GT}$ in order to determine the best one
 for truth discover over those claims.
\end{enumerate}

The steps 2--5 of the active learning algorithm are repeated until the stop criteria is satisfied. 
At the end of the active learning process, the truth discovery is finally realized by performing 
the union of the predictions of the ensemble of best truth finding algorithms for each kind of claims. 

