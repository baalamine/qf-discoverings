\section{About the Demonstration }\label{demonstration}



% Architecture and GUI
%\subsection{System Components}
%The architecture of our demonstration system, given in
%Figure~\ref{system_architecture}, comprises the following
%five main components.

% User I/O Interface
%\paragraph*{User Interface}
%The user interface is the main component that enables a given user  to interact with our system. Via the user interface component,  one has the ability to provide a query (or a key phrase), to recieve answers from the truth finding module or  label requests from the active ensembling module. When a label request is sent  is sent to the user, she (or he) also provides his answers through  this component.


% Information extraction module
%\paragraph*{OpenIE component} OpenIE is responsible to the extraction  of information from the Web. It considers, as an input, a user query and queries several Web corpus in order to returns the relevant set of candidate answers. Our OpenIE component relies on TextRunner engine for the extraction of the set of candidate Web claims with respect to a user query about a given real-world fact.

% Learning module
%\paragraph*{Active Ensembling Module} This is the core component of our system which discovers an optimal ensemble for truth finding over candidate claims extracted with the OpenIE  component. The active ensembling module, as described in Section~3, evaluates set of twelves truth finding algorithms on unlabeled claims, figures out the most controversal claims, and requests labels from the user. Once it obtained feebacks from the user, it estimates the accuracy of the competing algorithms on the labeled claims, and finally decides about the best one to use for each set of claims related  to the same query or fact. The module returns an ensemble since one can have distinct good truth finding techniques for various sets of claims about different facts.

% Truth finding engine
%\paragraph*{Truth Finding Module} The truth finding module uses the hypotheses found by the active ensembling  module about the best algorithm to use for each set input of claims to determine the final result that will maximize the precision of the truth finding discovery. Typically, the module combines the best of each algorithm in the ensemble returned by the learning process. It outputs its result to the user through the user interface component.



% knwoledge bases
%\paragraph*{Storage Module}The storage module consists of a local repository for the system
%that enables to have a dump of labeled claims. This set of labeled claims is a partial valuable
%ground truth. It can be used, if available, in order to boostrap our active learning process. For
%example as we already learnt about the best approach for these labeled claims, they can be compare 
%to new unlabeled claims, in terms of similar involved facts or similar conflict distribution, for 
%directly devising a candidate optimal truth discovery algorithm without having to run again the learning 
%procedure.

%\begin{figure}[ht]
%\begin{tikzpicture}[auto, line/.style ={draw,  thick, shorten <=0pt}]
 % \matrix[matrix of nodes, row sep=5ex, column sep=1em] (mx) {
  %  User I/O Interface & Truth Finding Module\\
   % IE Module & Ensembling Module \\
    %& Storage Module (labeled claims)\\ 
%  };
  %\node[ou=(mx-1-3)] (empty) {};
%  \node[ou=(mx-2-1)] (oi) {};
 % \node[ou=(mx-1-2)] (tf) {};
  %\node[ou=(mx-2-2)] (lm) {};
  %\node[ou=(mx-3-2)] (kb) {};
  %{[<-, thick]
  %\draw(mx-1-1)edge(tf);
%  }
 % {[->, thick]
  %\draw(mx-1-1)edge(oi);
 % \draw(oi)edge(lm);
%  }
%  {[<-, thick]
%    \draw(tf)edge(lm);
%  }
%  {[<->, thick]
%    \draw(lm)edge(kb);
%    \draw(mx-1-1)edge(lm);
%  }
 % \begin{scope}[every path/.style=line, <-]
  % \path(mx-1-1)  - | (tf);
  % \path(mx-1-3)  - | (lm);
  %\end{scope}

%\end{tikzpicture}
%\caption{System Components}\label{system_architecture}
%\end{figure}
% Demonstration scenario
%\subsection{Demonstration scenarios}
\laure{we need to find at least 2 "catchy" precise scenarios  we need to rework this section}
A given user that wants to interact with our system
must do it through the search form. Through the search
form, she (or he) provides her searched relation, e.g.,
``Where is born Barack Obama?".
The searched relation is then 
passed to the information extraction engine,  TextRunner system
in our case, which returns a set of answers considered to be 
relevant for the user's request. Each claim in the returned list is
processed in order to extract the corresponding sources along a detailed
description of the claim which we format in a certain manner. The 
set of sources and the formatted versions of all claims are then passed
to the truth finding module which integrate all the claims and compute the
most probable answer together with the reliability scores of participated 
sources for the searched relation. Finally, the output of the truth finding
process is returned to the user. The user can also want to review the output
of our system by definitively validiting it or not through its knwoledge of the
modeled world. For example when the system has totally wrong, it may be interesting
to get such a kind of feedbacks from the user in order to change the used method, as
there are many available with our system, and to enhance the process for the further 
search about the same world. The user gives feedbacks using the option buttons on the 
left-hand side of the outputted claims or the text form. The feebacks given by the user
is saved in knwoledge bases within our system for further processes.
