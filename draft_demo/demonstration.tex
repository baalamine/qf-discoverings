\begin{figure*}[t!]
 \begin{center}
  \includegraphics[width=.95\linewidth]{screenshot2.png}
   \end{center}
\label{screenshot}\caption{$\VERA$: Behind the Scene}
\end{figure*}

\section{demonstration scenario}
During the demo, we will show how $\VERA$  estimates the veracity
of multi-source information from  Web sources and tweets.
The audience will see a truth discovery scenario that can
not be accomplished using conventional search engines or existing
truth discovery methods and we will show how they can be handled
using $\VERA$.


\textbf{Fact-Checking for Crisis Situations.}
In crisis situations, time is critical when  an emergency response must be issued as soon as possible. Often the only information the public receives about the situation or the disaster is through the media (usually by authoritative sources) only once it is verified and but also immediately through social media as volunteered information that still needs to be checked. In this demonstration scenario,  $\VERA$ uses data from GDELT\footnote{{\scriptsize\url{http://www.gdeltproject.org/}}} and expands a tweet dataset obtained and classified using AIDR~\cite{AIDR} and estimates the veracity of  claims extracted from the content of tweets and Web source. Figure 2 presents  $\VERA$ behind the scene and shows the results of multiple truth discovery methods for various tragic events in 2015 (Paris bombing, Boston marathon explosion, Bamako hotel shooting, Nepal earth quake). Truth discovery methods have been applied to the claims extracted  by TwitIE and structured by $\VERA$ from a collection of tweets; the tweets were classified through AIDR in the category \textquote{injured or dead people}. As the time goes by, more information corroborate the true number of casualties. %$\VERA$ source view  also provides the trustworthiness scoring  and ranking of the sources (twitter ids and Web sources) evolving in time.


\textbf{Rumors.} Nowadays, rumors about facts related to persons (e.g., celebrities) or hot events are ubiquitous on the Web.
Some rumors are purposely propagated for misinformation or propaganda,  and others are tied to a certain context which requires to have more information as soon as possible to confirm or deny them (e.g., the rumor of the bombing of \textquote{Les Halles shopping center} during Paris attacks in November 2015). Such kinds of rumors often spread out very quickly in  social media due to the lack of effective means to detect them. This scenario will show how $\VERA$ operates on rumors  by leveraging the sources' trustworthiness and time-dependent consensus in truth discovery computation.

