%%% LaTeX Template
%%% This template can be used for both articles and reports.
%%%
%%% Copyright: http://www.howtotex.com/
%%% Date: February 2011

%%% Preamble
\documentclass[paper=a4, fontsize=11pt]{scrartcl}	% Article class of KOMA-script with 11pt font and a4 format

\usepackage[english]{babel}															% English language/hyphenation
\usepackage[protrusion=true,expansion=true]{microtype}				% Better typography
\usepackage{amsmath,amsfonts,amsthm}										% Math packages
\usepackage[pdftex]{graphicx}														% Enable pdflatex
%\usepackage{color,transparent}													% If you use color and/or transparency
\usepackage[hang, small,labelfont=bf,up,textfont=it,up]{caption}	% Custom captions under/above floats
\usepackage{epstopdf}																	% Converts .eps to .pdf
\usepackage{subfig}																		% Subfigures
\usepackage{booktabs}																	% Nicer tables


%%% Advanced verbatim environment
\usepackage{verbatim}
\usepackage{fancyvrb}
\DefineShortVerb{\|}								% delimiter to display inline verbatim text


%%% Custom sectioning (sectsty package)
\usepackage{sectsty}								% Custom sectioning (see below)
\allsectionsfont{%									% Change font of al section commands
	\usefont{OT1}{bch}{b}{n}%					% bch-b-n: CharterBT-Bold font
%	\hspace{15pt}%									% Uncomment for indentation
	}

\sectionfont{%										% Change font of \section command
	\usefont{OT1}{bch}{b}{n}%					% bch-b-n: CharterBT-Bold font
	\sectionrule{0pt}{0pt}{-5pt}{0.8pt}%	% Horizontal rule below section
	}


%%% Custom headers/footers (fancyhdr package)
\usepackage{fancyhdr}
\pagestyle{fancyplain}
\fancyhead{}														% No page header
\fancyfoot[C]{\thepage}										% Pagenumbering at center of footer
\fancyfoot[R]{\small }	% You can remove/edit this line 
\renewcommand{\headrulewidth}{0pt}				% Remove header underlines
\renewcommand{\footrulewidth}{0pt}				% Remove footer underlines
\setlength{\headheight}{13.6pt}

%%% Equation and float numbering
\numberwithin{equation}{section}															% Equationnumbering: section.eq#
\numberwithin{figure}{section}																% Figurenumbering: section.fig#
\numberwithin{table}{section}																% Tablenumbering: section.tab#


%%% Title	
\title{ \vspace{-1in} 	\usefont{OT1}{bch}{b}{n}
		\huge \strut Data Forensics with Analytics\strut \\
		\Large \bfseries \strut Trust Discovery for Data Quality\strut
}
\author{ 									\usefont{OT1}{bch}{m}{n}
        Mouhamadou Lamine Ba and Laure Berti-Equille\\		\usefont{OT1}{bch}{m}{n}
        Qatar Computing Research Institute\\	\usefont{OT1}{bch}{m}{n}
       % Random Department\\
        %\texttt{email@example.com}
}
\date{}

%%% Begin document
\begin{document}
\maketitle
 
 Data Forensics with Analytics, or DAFNA for short, is an ambitious project initiated by the Data Analytics Research Group in Qatar Computing Research
 Institute. It main goal is to provide effective algorithms and tools for determining the veracity of structured information and the reliability level 
 of data sources. Being able to efficiently verify the veracity of data and sources in presence is an ubiquitous challenge in many real world scneraios,
 e.g., data fusion or social data analysis, in which human has a need to consume high quality information for personal or businness purposes. Therefore,
 DAFNA's vision is to fill the gap about information veracity management in real applications, in particular those related to everyday life in Qatar.
 This is challenging and asks for facing diverses research topics related to efficient fact
 extraction, efficient truth finding strategies, and how truth finders could be coupled with real existing systems.

 
 
 We will present our ongoing study on extensively comparing twelve existing truth discovery algorithms, releasing the first API that will 
 enable user applications to transparently access to the truth finders, and finally integrating truth finding in the process of quality information 
 retrieval based on Web extraction systems.
 
 \paragraph*{Roadmap of Truth Discovery Algorithms}The first challenge dealt by the DAFNA project was to provide an thorough roadmap of the existing 
 truth discovery algorithms with an extensive comparative evaluation of their performance.
 
 
 \paragraph*{RestFul API for 3-tier Applications}
 \paragraph*{Real Applications in Qatar}
\end{document}