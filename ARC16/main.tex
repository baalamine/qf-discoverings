%%% LaTeX Template
%%% This template can be used for both articles and reports.
%%%
%%% Copyright: http://www.howtotex.com/
%%% Date: February 2011

%%% Preamble
\documentclass[paper=a4, fontsize=11pt]{scrartcl}	% Article class of KOMA-script with 11pt font and a4 format

\usepackage[english]{babel}															% English language/hyphenation
\usepackage[protrusion=true,expansion=true]{microtype}				% Better typography
\usepackage{amsmath,amsfonts,amsthm}										% Math packages
\usepackage[pdftex]{graphicx}														% Enable pdflatex
%\usepackage{color,transparent}													% If you use color and/or transparency
\usepackage[hang, small,labelfont=bf,up,textfont=it,up]{caption}	% Custom captions under/above floats
\usepackage{epstopdf}																	% Converts .eps to .pdf
\usepackage{subfig}																		% Subfigures
\usepackage{booktabs}																	% Nicer tables


%%% Advanced verbatim environment
\usepackage{verbatim}
\usepackage{fancyvrb}
\DefineShortVerb{\|}								% delimiter to display inline verbatim text


%%% Custom sectioning (sectsty package)
\usepackage{sectsty}								% Custom sectioning (see below)
\allsectionsfont{%									% Change font of al section commands
	\usefont{OT1}{bch}{b}{n}%					% bch-b-n: CharterBT-Bold font
%	\hspace{15pt}%									% Uncomment for indentation
	}

\sectionfont{%										% Change font of \section command
	\usefont{OT1}{bch}{b}{n}%					% bch-b-n: CharterBT-Bold font
	\sectionrule{0pt}{0pt}{-5pt}{0.8pt}%	% Horizontal rule below section
	}


%%% Custom headers/footers (fancyhdr package)
\usepackage{fancyhdr}
\pagestyle{fancyplain}
\fancyhead{}														% No page header
\fancyfoot[C]{\thepage}										% Pagenumbering at center of footer
\fancyfoot[R]{\small }	% You can remove/edit this line 
\renewcommand{\headrulewidth}{0pt}				% Remove header underlines
\renewcommand{\footrulewidth}{0pt}				% Remove footer underlines
\setlength{\headheight}{13.6pt}

%%% Equation and float numbering
\numberwithin{equation}{section}															% Equationnumbering: section.eq#
\numberwithin{figure}{section}																% Figurenumbering: section.fig#
\numberwithin{table}{section}																% Tablenumbering: section.tab#


%%% Title	
\title{ \vspace{-1in} 	\usefont{OT1}{bch}{b}{n}
		\huge \strut Data Forensics with Analytics\strut \\
		\Large \bfseries \strut Trust Discovery for Data Quality\strut
}
\author{ 									\usefont{OT1}{bch}{m}{n}
        Mouhamadou Lamine Ba and Laure Berti-Equille\\		\usefont{OT1}{bch}{m}{n}
        Qatar Computing Research Institute\\	\usefont{OT1}{bch}{m}{n}
       % Random Department\\
        %\texttt{email@example.com}
}
\date{}

%%% Begin document
\begin{document}
\maketitle
 
 Data Forensics with Analytics, or DAFNA for short, is an ambitious project initiated within the Data Analytics Research Group in Qatar Computing Research
 Institute. It main goal is to provide effective algorithms and tools for determining the veracity of structured information when they originate from  
 multiple untrustworthy sources. The ability to efficiently estimate the veracity of data, along with the reliability level of the sources in presence,
 is an ubiquitous problem in many real world use cases, e.g., data fusion or social data analysis, in which human rely on an automated data acquisition
 and integration process in order to consume high quality information for personal or business purposes. DAFNA's vision is to fill the gap of the  lack of 
 a comprehensive framework for information veracity management, with a direct impact on applications related to Qatar. Such a challenge requires to investigate
 various research topics such as a general purpose truth discovery algorithm and its applicability in practice.

We will present our ongoing study on extensively comparing the state-of-the-art truth discovery algorithms, releasing the first REST API for truth discovery algorithms,
and designing an hybrid truth discovery approach using active ensembling. Finally, we will briefly present real applications of truth discovery in Qatar.
.
 
 \paragraph*{Efficient Truth Discovery}Truth discovery is a hard problem to deal with in practical as often there is no a-priori
 about the veracity of provided information and the reliabily level of the sources. 
 The first challenge dealt by the DAFNA project was to provide a comprehensible study of existing
 truth discovery algorithms. 
 
 \begin{itemize}
  \item Comparative Study of existing truth discovery algorithms
  \item Ensembling based active learning strategy 
  \item Design and release of a RestFul API for third party applications
 \end{itemize}
 \paragraph*{Potential Applications in Qatar}Information about Qatar are now available on multiple places. General 
 purposes Web applications 
\end{document}