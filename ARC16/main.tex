%%% LaTeX Template
%%% This template can be used for both articles and reports.
%%%
%%% Copyright: http://www.howtotex.com/
%%% Date: February 2011

%%% Preamble
\documentclass[paper=a4, fontsize=11pt]{scrartcl}	% Article class of KOMA-script with 11pt font and a4 format

\usepackage[english]{babel}															% English language/hyphenation
\usepackage[protrusion=true,expansion=true]{microtype}				% Better typography
\usepackage{amsmath,amsfonts,amsthm}										% Math packages
\usepackage[pdftex]{graphicx}														% Enable pdflatex
%\usepackage{color,transparent}													% If you use color and/or transparency
\usepackage[hang, small,labelfont=bf,up,textfont=it,up]{caption}	% Custom captions under/above floats
\usepackage{epstopdf}																	% Converts .eps to .pdf
\usepackage{subfig}																		% Subfigures
\usepackage{booktabs}																	% Nicer tables
\usepackage{mathptmx}
\usepackage{paralist}
\usepackage{hyperref}
%\usepackage{crunch}

%%% Advanced verbatim environment
\usepackage{verbatim}
\usepackage{fancyvrb}
\DefineShortVerb{\|}								% delimiter to display inline verbatim text


%%% Custom sectioning (sectsty package)
\usepackage{sectsty}								% Custom sectioning (see below)
\allsectionsfont{%									% Change font of al section commands
	\usefont{OT1}{bch}{b}{n}%					% bch-b-n: CharterBT-Bold font
%	\hspace{15pt}%									% Uncomment for indentation
	}

\sectionfont{%										% Change font of \section command
	\usefont{OT1}{bch}{b}{n}%					% bch-b-n: CharterBT-Bold font
	\sectionrule{0pt}{0pt}{-5pt}{0.8pt}%	% Horizontal rule below section
	}


%%% Custom headers/footers (fancyhdr package)
%\usepackage{fancyhdr}
%\pagestyle{fancyplain}
%\fancyhead{}														% No page header
%\fancyfoot[C]{\thepage}										% Pagenumbering at center of footer
%\fancyfoot[R]{\small }	% You can remove/edit this line 
%\renewcommand{\headrulewidth}{0pt}				% Remove header underlines
%\renewcommand{\footrulewidth}{0pt}				% Remove footer underlines
%\setlength{\headheight}{13.6pt}

%%% Equation and float numbering
\numberwithin{equation}{section}															% Equationnumbering: section.eq#
\numberwithin{figure}{section}																% Figurenumbering: section.fig#
\numberwithin{table}{section}																% Tablenumbering: section.tab#


%%% Title	
\title{ \vspace{-1in} 	\usefont{OT1}{bch}{b}{n}
		\huge \strut Discovering the truth in the Web Data\strut \\
		\Large \bfseries \strut One Facet of Data Forensics\strut
}
\author{ 									\usefont{OT1}{bch}{m}{n}
        Mouhamadou Lamine Ba, Laure Berti-Equille, Hossam M. Hammady\\		\usefont{OT1}{bch}{m}{n}
        Qatar Computing Research Institute, Hamad Bin Khalifa University\\	\usefont{OT1}{bch}{m}{n}
       % Random Department\\
        %\texttt{email@example.com}
}
\date{}

%%% Begin document
\begin{document}
\maketitle
 \vspace*{-1.5cm}
 Data Forensics with Analytics, or DAFNA for short, is an ambitious project initiated by the Data Analytics Research Group in Qatar Computing Research
 Institute, Hamad Bin Khalifa University. Its main goal is to provide effective algorithms and tools for determining the veracity of structured information when they originate
 from  multiple sources. The ability to efficiently estimate the veracity of data, along with the reliability level of the sources in presence, is a challenging
 problem in many real world use cases (e.g., data fusion, social data analytics, rumor detection, etc.) in which users rely on a semi-automatic data extraction
 and integration process in order to consume high quality information for personal or business purposes. DAFNA's vision is to provide a suite of tools for Data
 Forensics and investigates various research topics such as fact-checking and truth discovery and their practical applicability.

We will present our ongoing development (\url{dafna.qcri.org}) on extensively comparing the state-of-the-art truth discovery algorithms, releasing a new system and 
the first REST API for truth discovery algorithms, and designing an hybrid truth discovery approach using active ensembling. Finally, we will briefly present real-world
applications of truth discovery from the Web data.
 
 \paragraph*{Efficient Truth Discovery}Truth discovery is a hard problem to deal with in practical since often there is no a priori knowledge about the veracity of provided 
 information and the reliabily level of the sources. This raises many questions about a thorough understanding of the state-of-the-art truth discovery algorithms and their applicability
 for \emph{actionable} truth discovery. A new truth discovery approach is needed and should be rather comprehensible and domain independent. In addition, it should take advantage of the 
 benefits of existing solutions, while being built on realistic assumptions for an easy use in real applications. In this context, we propose a study to deal with open truth discovery challenges 
 and consists of the following contributions:
 \begin{inparaenum}[(i)]
  \item The thorough comparative study of existing truth discovery algorithms;
  \item The design and release of the first online truth discovery system and 
  the first REST API for truth discovery available at \url{dafna.qcri.org};
  \item An hybrid truth discovery approach using active ensembling; and 
  \item An application to query answering related to Qatar where the veracity of information provided by multiple Web sources is estimated.
 \end{inparaenum}
 \cite{*}
 \bibliographystyle{plain}
 \bibliography{biblio}
\end{document}